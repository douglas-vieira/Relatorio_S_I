\documentclass[a4paper, 12pt]{article}
\usepackage[portuguese]{babel}
\usepackage[utf8]{inputenc}
\usepackage[T1]{fontenc}
\usepackage{amssymb}
\usepackage{indentfirst}
\usepackage{graphicx}
\usepackage{fancyhdr}
\usepackage{circuitikz}
\usepackage{tikz}
\usepackage{authblk}
\usepackage{geometry}
\usepackage[colorinlistoftodos]{todonotes}
\usepackage{caption}
\geometry{a4paper, total={170mm,257mm}, left=25mm, top=25mm, right=25mm, bottom=25mm}
\usetikzlibrary{positioning}
\DeclareGraphicsExtensions{.pdf,.png,.jpg}
\makeindex
\author[]{Douglas Alcantara - 140038329}
\affil[]{Sisttemas de Informação\\Universidade de Brasília}
\title{Tecnical Report - Big Data (analytics, applications) }
\date{29 de Abril de 2016}
\pagestyle{fancy}
\fancyhf{}
\lhead{4º Experimento}
\rhead{Universidade de Brasília\\IE - Departamento de Ciência da Computação}
\headsep = 40pt
\captionsetup{labelformat=empty}
\begin{document}
	\begin{titlepage}

		\newcommand{\HRule}{\rule{\linewidth}{0.5mm}}
		\centering
        %\maketitle
	    \textsc{\LARGE Universidade de Brasília}\\[0.5cm]
	    \includegraphics{logo.jpg}\\[0.5cm]
		\textsc{\Large Instituto de Ciências Exatas}\\[0.5cm]
	    \textsc{\Large Departamento de Ciência da Computação}\\[0.5cm]
		\textsc{\Large Sistemas de Informação}\\[0.5cm]
	    \HRule \\[0.4cm]
	    { \huge \bfseries Big Data (analytics, applications)}\\
	      \HRule \\[1.5cm]
	      	\begin{minipage}{0.4\textwidth}
	      		\begin{flushleft} \large
	      			\emph{Nome:}\\
	     			\emph{Douglas Alcantara}\\
	      		\end{flushleft}
	      	\end{minipage}
	      	~
	      	\begin{minipage}{0.4\textwidth}
	      		\begin{flushright} \large
	      			\emph{Matrícula:}\\
	     			\textsc{14/0038329}\\
	      		\end{flushright}
	      	\end{minipage}\\[2cm]
		\textsc{\large \centering 29 de Abril de 2016}\\
	\end{titlepage}
	\newpage
	\textrm{
				• O que é?\\
				• Quem definiu?\\
				• Qual o histórico?\\
				• Para que serve?\\
				• Que tipo de problema pode ser resolvido?\\
				• Quais as tecnologias, métodos, ferramentas (proprietária, softw livre)\\
			}
\end{document}
